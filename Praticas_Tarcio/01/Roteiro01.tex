\documentclass[a4paper,12pt]{article}

% Including essential packages for document structure and formatting
\usepackage[utf8]{inputenc}
\usepackage[T1]{fontenc}
\usepackage{lmodern}
\usepackage[portuguese]{babel}
\usepackage{geometry}
\geometry{margin=2.5cm}
\usepackage{parskip}
\usepackage{enumitem}
\usepackage{hyperref}
\usepackage{fancyhdr}
\usepackage{lastpage}

% Setting up header and footer
\pagestyle{fancy}
\fancyhf{}
\fancyhead[L]{Prática 1: Piscar LED}
\fancyfoot[C]{\thepage\ de \pageref{LastPage}}

% Configuring fonts
\usepackage{noto}

% Customizing hyperlink style
\hypersetup{
    colorlinks=true,
    linkcolor=blue,
    urlcolor=blue
}

\begin{document}

% Creating title section
\begin{center}
    \textbf{\Large Prática 1: Piscar LED (Ref: Livro - Proj. 1, Cap. 2)}
\end{center}

\vspace{1cm}

% Outlining lesson plan
\section*{Roteiro de Aula}

\begin{itemize}
    \item \textbf{Título da Aula:} Introdução ao Arduino: Fazendo um LED Piscar
    \item \textbf{Duração Estimada:} 50-60 minutos
    \item \textbf{Objetivos de Aprendizagem:}
    \begin{itemize}
        \item Identificar os componentes básicos: Arduino Uno, LED, resistor, protoboard, jumpers.
        \item Montar um circuito simples em uma protoboard seguindo um diagrama.
        \item Compreender a estrutura básica de um código Arduino (\texttt{setup()} e \texttt{loop()}).
        \item Utilizar comandos básicos: \texttt{pinMode()}, \texttt{digitalWrite()}, \texttt{delay()}.
        \item Fazer o upload de um código para o Arduino (ou simular no Tinkercad).
        \item Entender a função do resistor para proteger o LED.
    \end{itemize}
    \item \textbf{Materiais:}
    \begin{itemize}
        \item Computador com Arduino IDE instalado OU acesso ao Tinkercad.
        \item Kit Arduino: Placa Arduino Uno, cabo USB, protoboard, 1 LED (qualquer cor), 1 resistor (220$\Omega$ ou 330$\Omega$), fios jumper.
        \item (Opcional) Projetor.
    \end{itemize}
    \item \textbf{Procedimento:}
    \begin{enumerate}
        \item \textbf{Introdução (5 min):} Apresentar Arduino, placa Uno, objetivo da aula.
        \item \textbf{Componentes e Protoboard (10 min):} Apresentar LED (polaridade), Resistor (função), Protoboard (conexões), Jumpers.
        \item \textbf{Montagem do Circuito (10 min):} Exibir diagrama, demonstrar montagem (física ou Tinkercad), auxiliar alunos.
        \item \textbf{Introdução ao Código (15 min):} Abrir IDE/Tinkercad, explicar \texttt{setup()}/\texttt{loop()}, explicar código linha a linha (\texttt{int}, \texttt{pinMode}, \texttt{digitalWrite}, \texttt{delay}, comentários).
        \item \textbf{Execução e Teste (5 min):} Mostrar verificação/upload/simulação, observar LED, resolver problemas.
        \item \textbf{Discussão e Variações (5 min):} Alterar \texttt{delay()}, recapitular comandos.
    \end{enumerate}
    \item \textbf{Avaliação:} Observação da montagem, funcionamento do código, perguntas sobre comandos.
    \item \textbf{Desafios (Opcional/Tarefa):} Alterar padrão de piscada, adicionar segundo LED.
\end{itemize}

% Adding Tinkercad link
\vspace{0.5cm}
\noindent
\textbf{Link:} \href{https://www.tinkercad.com/things/78oGplaYCtD-pratica-1-piscar-led-ref-livro-proj-1-cap-2}{Tinkercad - Simulação do Circuito}

\end{document}