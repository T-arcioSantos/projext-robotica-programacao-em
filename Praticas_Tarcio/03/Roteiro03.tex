\documentclass[a4paper,12pt]{article}

% Pacotes essenciais para estrutura e formatação
\usepackage[utf8]{inputenc}
\usepackage[T1]{fontenc}
\usepackage{lmodern}
\usepackage[portuguese]{babel}
\usepackage{geometry}
\geometry{margin=2.5cm} % Margens adequadas
\usepackage{parskip} % Espaçamento entre parágrafos
\usepackage{enumitem} % Para listas personalizadas
\usepackage{hyperref} % Para links clicáveis
\usepackage{fancyhdr} % Para cabeçalhos e rodapés
\usepackage{lastpage} % Para referenciar a última página no rodapé

% Configuração de cabeçalho e rodapé
\pagestyle{fancy}
\fancyhf{} % Limpa cabeçalho e rodapé atuais
\fancyhead[L]{Prática 2: Exibir Texto no LCD 16x2} % Título da prática no cabeçalho esquerdo
\fancyfoot[C]{\thepage\ de \pageref{LastPage}} % Numeração de página no centro do rodapé

% Configuração de fontes (Noto é uma boa opção moderna)
\usepackage{noto}

% Personalização do estilo dos hyperlinks
\hypersetup{
    colorlinks=true,
    linkcolor=blue,
    urlcolor=blue
}

\begin{document}

% Seção do Título
\begin{center}
    % Ajuste a referência se houver um capítulo/projeto específico no livro
    \textbf{\Large Prática 2: Exibir Texto no Display LCD 16x2}
    % \textbf{\Large (Ref: Livro - Proj. X, Cap. Y)} % Descomente e ajuste se necessário
\end{center}

\vspace{1cm}

% Roteiro da Aula
\section*{Roteiro de Aula}

\begin{itemize}
    \item \textbf{Título da Aula:} Introdução ao Display LCD 16x2 com Arduino
    \item \textbf{Duração Estimada:} 60-75 minutos (montagem mais complexa)
    \item \textbf{Objetivos de Aprendizagem:}
    \begin{itemize}
        \item Identificar os componentes: Arduino Uno, Display LCD 16x2, Potenciômetro, Resistor, Protoboard, Jumpers.
        \item Compreender a função dos principais pinos do LCD (VCC, GND, VO, RS, E, D4-D7, A, K).
        \item Montar o circuito do LCD na protoboard seguindo um diagrama.
        \item Entender a necessidade e o uso do potenciômetro para ajustar o contraste.
        \item Entender a função do resistor para o backlight do LCD.
        \item Incluir e utilizar a biblioteca \texttt{LiquidCrystal.h}.
        \item Compreender a estrutura básica do código para LCD: inicialização, posicionamento e escrita.
        \item Utilizar comandos essenciais: \texttt{LiquidCrystal()}, \texttt{lcd.begin()}, \texttt{lcd.setCursor()}, \texttt{lcd.print()}, \texttt{lcd.clear()}.
        \item Fazer o upload do código para o Arduino (ou simular no Tinkercad) e visualizar o texto.
        \item Realizar o ajuste de contraste no display físico ou simulado.
    \end{itemize}
    \item \textbf{Materiais:}
    \begin{itemize}
        \item Computador com Arduino IDE instalado OU acesso ao Tinkercad.
        \item Kit Arduino: Placa Arduino Uno, cabo USB, protoboard.
        \item \textbf{Display LCD 16x2} (com pinos soldados, se físico).
        \item \textbf{Potenciômetro de 10k$\Omega$}.
        \item 1 Resistor (220$\Omega$ ou 330$\Omega$) para o backlight.
        \item Fios jumper (Macho-Macho).
        \item (Opcional) Projetor.
    \end{itemize}
    \item \textbf{Procedimento:}
    \begin{enumerate}
        \item \textbf{Introdução (5 min):} Apresentar o display LCD, suas aplicações (mostrar texto/dados), objetivo da aula.
        \item \textbf{Componentes e Pinos (15 min):} Explicar o LCD 16x2, função dos pinos essenciais (alimentação, contraste, controle, dados, backlight). Apresentar o potenciômetro (ajuste de contraste) e o resistor (proteção do backlight). Revisar protoboard.
        \item \textbf{Montagem do Circuito (15-20 min):} Exibir diagrama de conexão claro. Demonstrar a montagem passo a passo (física ou Tinkercad), enfatizando a atenção aos pinos corretos. Auxiliar os alunos (esta etapa exige mais cuidado).
        \item \textbf{Introdução ao Código (15 min):}
            \begin{itemize}
                \item Abrir IDE/Tinkercad.
                \item Explicar a necessidade da biblioteca: \texttt{#include <LiquidCrystal.h>}.
                \item Explicar a inicialização do objeto LCD: \texttt{LiquidCrystal lcd(rs, en, d4, d5, d6, d7);}, relacionando com os pinos do Arduino usados.
                \item No \texttt{setup()}: \texttt{lcd.begin(16, 2);} para definir o tamanho.
                \item Comandos básicos: \texttt{lcd.print("Texto");}, \texttt{lcd.setCursor(col, lin);}, \texttt{lcd.clear();}.
                \item Mostrar o código de exemplo completo.
            \end{itemize}
        \item \textbf{Execução e Teste (10 min):} Mostrar verificação/upload/simulação. \textbf{Demonstrar o ajuste do contraste usando o potenciômetro} até o texto ficar nítido. Observar o backlight. Solucionar problemas comuns (conexões erradas, contraste mal ajustado).
        \item \textbf{Discussão e Variações (5 min):} Como limpar a tela (\texttt{lcd.clear()}), como escrever em locais diferentes (\texttt{setCursor}), como exibir números ou variáveis.
    \end{enumerate}
    \item \textbf{Avaliação:} Observação da montagem correta, funcionamento do código, capacidade de ajustar o contraste, perguntas sobre os comandos e a função dos componentes.
    \item \textbf{Desafios (Opcional/Tarefa):}
        \begin{itemize}
            \item Fazer o texto rolar na tela.
            \item Criar um contador que atualiza no LCD.
            \item Exibir dados de um sensor (se disponível/próxima aula).
            % \item (Avançado) Criar caracteres customizados.
        \end{itemize}
\end{itemize}

% Adicionar link para o Tinkercad (substitua pela URL correta)
\vspace{0.5cm}
\noindent
% \textbf{Link para Simulação:} \href{URL_DO_SEU_PROJETO_TINKERCAD_AQUI}{Tinkercad - Simulação LCD 16x2}
\textbf{Link para Simulação (Exemplo - Crie o seu):} \url{https://www.tinkercad.com/} (Lembre-se de criar e compartilhar o link do seu projeto específico)

\end{document}